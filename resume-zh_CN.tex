% !TEX TS-program = xelatex
% !TEX encoding = UTF-8 Unicode
% !Mode:: "TeX:UTF-8"
\documentclass{resume}
\usepackage{zh_CN-Adobefonts_external} % Simplified Chinese Support using external fonts (./fonts/zh_CN-Adobe/)
%\usepackage{zh_CN-Adobefonts_internal} % Simplified Chinese Support using system fonts
\usepackage{linespacing_fix} % disable extra space before next section
\usepackage{cite}
\usepackage[colorlinks,linkcolor=blue,anchorcolor=blue,citecolor=green,urlcolor=blue]{hyperref}
\AtBeginDocument{%
\linespread{1.05}\selectfont
}

\begin{document}
\pagenumbering{gobble} % suppress displaying page number

\name{ 沈 \hspace{0.1cm}登 \hspace{0.1cm}年 }

% {E-mail}{mobilephone}{homepage}
% be careful of _ in emaill address
\contactInfo{\href{mailto:lovesmilegod@qq.com}{lovesmilegod@qq.com}}{15099738604}{}
% {E-mail}{mobilephone}
% keep the last empty braces!
%\contactInfo{xxx@yuanbin.me}{(+86) 131-221-87xxx}{}
\textbf{求职意向}: 前端工程师/深圳
\section{\faGraduationCap 教育背景}
\datedsubsection{\textbf{中国北京石油大学(非统招)}}{2016 年 9 月 – 2018 年}

%\textit{Ph.D Candidate} in Computer Science
%
%Adviser: Prof.Qiang Liu \ \

\datedsubsection{\textbf{在读专科 计算机科学与技术}}{}
%

\section{\faCogs\ IT 技能}
% increase linespacing [parsep=0.5ex]
\begin{itemize}[parsep=0.5ex]
  \item 研究兴趣: 前端开发 / 掘金 / 知乎 / Github
  \item 编程语言: HTML / CSS  / Javascript / AJAX
  \item 编程框架/库: Bootstrap / jQuery /  Vue / React
  \item 编辑软件: WebStorm / Photoshop /  LAMP
\end{itemize}


\section{\faUsers\ 个人经历}


\datedsubsection{\textbf{岗位及职责}:深圳世金宝外贸有限公司产品经理}{2014 年 10月 – 2016年 10 月}
\role{}{负责公司热敏电阻整体规划,整理产品文档,产品原型,交互流程,撰写产品文档,负责行业观察,竞品分析,为公司定期输出分析报告 根据市场调研,用户反馈及数据分析结果,持续优化公司电子类产品}
\role{}{协助销售部制定多种规格产品的销售方案,与研发部探讨新产品原型制定方案并在芯片研发取得较大突破,致主要产品的精度和稳定性大幅度提升,提高了企业更强的市场竞争能力
}
\datedsubsection{\textbf{自主学习}:}{2016 年 8 月 – 2017 年 6 月}
\role{}{学习W3c标准编码规范,运用Bootstrap 实现能够兼容多个终端的响应式静态网页布局与原生javascript 操作 DOM 的学习编码. 学习HTML/XHTML,  CSS, Javascript, AJAX , JSON 前端技术.深入理解 this, 闭包, 原型, 继承, 作用域等概念,
了解使用jQuery库, Bootstrap框架, Vue.js框架进行开发.
学习Git多人协作开发}
% \begin{itemize}
% 	\item Anayzing outlier points in PGData.
% 	\item Using GPU to accelerate Image Clustering for ARShop Project.
% \end{itemize}
\datedsubsection{\textbf{项目描述}:}{2017 年 5 月 – 2017 年 10月}
\role{}{在码市里远程与他人协作开发网址导航网站,该网址导航架构基于java平台开发,后台页面每一个板块拥有相应的配置控制,前端有换肤功能,能够切换两种以上风格导航。}
\datedsubsection{\textbf{项目职责}:}{}
\role{}{用javascript与jQuery实现交互逻辑与后端进行数据对接,用HTML和CSS开发符合w3c标准并兼容多种浏览器的网站页面,项目周期为60天}
\role{}{参考慕课网饿了么教程独自用所学综合知识与Vue.js框架配合开发出一款类似饿了么的网页APP}
%\datedsubsection{\textbf{Sesame Centre \& DataBase Group}, 新加坡国立大学, 新加坡}{2016年4月 -- 2016年8月}
%\role{研究实习生}{导师: \ Prof.Anthony K. H. Tung(邓锦浩)}
%\textbf{科研主题}: 用GPU进行图片聚类优化
%
%
%
%
%\datedsubsection{\textbf{ArticuLab}, 卡内基梅隆大学, 美国}{2015年7月 -- 2015年9月}
%\role{暑期科研实习生}{导师:\ Prof.Justine Cassell \& Dr.Alexandros Papangelis}
%\textbf{科研主题}:研究用多模态数据对人与人、人与虚拟机器人的关系建模
%% \begin{itemize}
%% \item 用openSMILE 和 CLM-Framework 提取了音频特征和图像特征,并将其同步化
%% \item 解决了清理数据、特征筛选、不平衡数据等常见问题,用Weka完成基础算法实验
%% \item 用线性条件随机场(linear-CRF)将该问题描述成时序问题,实现74\%的准确率
%% \item 用隐式条件随机场(Hidden CRF)来描绘关系之间的内在联系,实现70\%的准确率
%% \end{itemize}
%
%\datedsubsection{\textbf{机器感知与智能教育部重点实验室}, 北京大学, 中国}{2014年10月 -- 2016年4月}
%\role{研究助教}{导师: \ 宋国杰教授}
%\textbf{科研主题}:移动社交网络中的关系识别
%% \begin{itemize}
%% \item 提出一种基于\textbf{TF-IDF}, 可以从LBS服务(如百度地图API)中提取地理语意的方法
%% \item 在社交网络图中用了一种平衡敏感参数,改善了数据不平衡问题
%% \item 在移动通话网络中使用社交学中的结构平衡理论在模型中建立了三元团因子
%% \item 基于以上发现构建了一个基于因子图的模型(Balanced Triadic Factor Graph)
%% \end{itemize}
%
%\datedsubsection{\textbf{Mathematical Modeling \& Algorithms Laboratory}, 筑波大学, 日本}{2014年7月}
%\role{访问学者}{导师:前田恭行博士}
%\textbf{科研主题}:图像识别科训练
%% \begin{itemize}
%% \item 学习使用了一些基础机器学习方法(SVM,PCA)等来解决图像识别问题
%% \item 使用PCA, SVD方法解决了简单的手写数字图像识别问题
%% \end{itemize}
%
%\datedsubsection{\textbf{研发部},职圈科技有限公司,北京}{2015年2月 -- 2015年6月}
%\role{研发工程师}{经理:吕云}
%\textbf{项目}:职圈安卓手机App
%% \begin{itemize}
%% \item 独立编写完成了App社交及通讯板块功能
%% \item 使用\textbf{User-Based}算法初步实现了用户推荐板块
%% \end{itemize}
%
%
%% Reference Test
%%\datedsubsection{\textbf{Paper Title\cite{zaharia2012resilient}}}{May. 2015}
%%An xxx optimized for xxx\cite{verma2015large}
%%\begin{itemize}
%%  \item main contribution
%%\end{itemize}

\section{\faHeartO\ 前端技能}
%\begin{itemize}
%\item
\begin{itemize}
\item{熟悉w3c标准,对表现与数据分离 代码规范 web语义化有比较深刻的理解}
\item{熟练HTML/XHTML、 CSS、 Javascript 、AJAX 、JSON 前端技术 }
\item{了解善于运用jQuery库、Bootstrap框架、Vue.js框架进行开发}
\item{熟练Git及Github的使用}
\end{itemize}
%\end{itemize}

\section{\faPaperPlane\ 其他信息}

\datedline{\textbf{个人博客}:\ \  \url{http://www.shendengnian.com}}{}
\datedline{\textbf{Github 主页}: \ \ \url{https://github.com/shendengnian}}{}



%% Reference
%\newpage
%\bibliographystyle{IEEETran}
%\bibliography{mycite}
\end{document}
